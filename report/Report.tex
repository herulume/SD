\documentclass[a4paper]{article}

\usepackage[utf8]{inputenc}
\usepackage[portuguese]{babel}
\usepackage{graphicx}
\usepackage{a4wide}
\usepackage[pdftex,hidelinks]{hyperref}
\usepackage{float}
\usepackage{indentfirst}
\usepackage{subcaption}
\usepackage[cache=false]{minted}

\begin{document}


\title{Sistemas Distribuídos\\ Grupo 16 \\ Alocação de servidores na nuvem}
\author{Eduardo Barbosa (a83344) \and Bárbara Cardoso (a80453) \and Pedro Mendes (a79003)}
\date{\today}

\begin{titlepage}

    %título
    \thispagestyle{empty}
    \begin{center}
        \begin{minipage}{0.75\linewidth}
            \centering
            %engenharia logo
            \includegraphics[width=0.4\textwidth]{eng.jpeg}\par\vspace{1cm}
            \vspace{1.5cm}
            %títulos
            \href{https://www.uminho.pt/PT}{\scshape\LARGE Universidade do Minho} \par
            \vspace{1cm}
            \href{https://www.di.uminho.pt/}{\scshape\Large Departamento de Informática} \par
            \vspace{1.5cm}

            \maketitle
        \end{minipage}
    \end{center}

\end{titlepage}

\tableofcontents

\pagebreak

\section{Introdução}

\pagebreak
\section{Descrição do Problema}

\pagebreak
\section{Resolução do Problema}

Para a implementação deste sistema foi criada uma classe \texttt{Client}, que funciona de forma muito similar ao comando \texttt{telnet}, para que fosse possível interagir com o servidor.

A lógica do servidor está dividida em 3 \textit{packages}:

\begin{figure}[H]
    \begin{tabular}{ll}
        \texttt{server}            & que contem a lógica de negócio.\\
        \texttt{server/middleware} & que contem a \texttt{Session}.\\
        \texttt{server/exceptions} & que contem as \textit{exceptions} lançadas pela lógica de negócio.\\
        \texttt{util}              & que contem classes utilitárias.
    \end{tabular}
\end{figure}

\subsection{server.middleware}
\subsubsection{\texttt{Session}}

% Notes for this section:
% Session == Controller if considering MVC

\subsection{server}

% Notes for this section:
% server.* == Model if considering MVC

\subsection{util}
\subsubsection{\texttt{AtomicInt}}

Esta classe implementa uma versão \textit{thread safe} de um \texttt{Integer} disponibilizando uma API restrita que apenas permite alterações atómicas do seu valor. Esta é útil para garantir que os IDs auto-incrementados dos \texttt{Droplet}s nunca se repetem, caso dois fossem instanciados em simultâneo.

\subsubsection{\texttt{ThreadSafeMap}}

Esta implementação da interface \texttt{Map<K,V>} garante que todas as operações feitas sobre ele são atómicas e sequenciais, utilizando um \textit{Read Write Lock} para que possam haver leituras em simultâneo mas nunca escritas e/ou leituras em simultâneo.

\subsubsection{\texttt{ThreadSafeMutMap}}

Esta extensão do \texttt{ThreadSafeMap} é mais restrita quanto aos objectos que permite guardar, estes tem de implementar a interface \texttt{Lockable} que define objectos com a possibilidade de ser bloqueados por uma \textit{thread}. Esta restrição é necessária para ser possível codificar o seguinte padrão.

\begin{figure}[H]
    \begin{minted}{java}
        this.lock.lock();
        V v = this.map.get(k);
        v.lock();
        this.lock.unlock();
        return v;
    \end{minted}
\end{figure}

Assim a API desta estrutura de dados disponibiliza os seguintes métodos para que seja explicito que o objecto retornado esta \textit{locked} e terá de ser \textit{unlocked} para que não sejam criadas situações de \textit{deadlock}.

\begin{itemize}
    \item \texttt{public V getLocked(K k)}
    \item \texttt{public Collection<V> getLocked(K k)}
    \item \texttt{public V putLocked(K k, V v)}
    \item \texttt{public Collection<V> valuesLocked()}
    \item \texttt{public Set<Entry<K, V>> entrySetLocked()}
\end{itemize}

\pagebreak
\section{Conclusões e Trabalho Futuro}


\end{document}

