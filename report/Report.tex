\documentclass[a4paper]{article}

\usepackage[utf8]{inputenc}
\usepackage[portuguese]{babel}
\usepackage{graphicx}
\usepackage{a4wide}
\usepackage[pdftex,hidelinks]{hyperref}
\usepackage{float}
\usepackage{indentfirst}
\usepackage{subcaption}

\begin{document}

\renewcommand\thesection{}
\renewcommand\thesubsection{}
\renewcommand\thesubsubsection{}

\title{Sistemas Distribuídos\\ Grupo 16 \\ Alocação de servidores na nuvem}
\author{Eduardo Barbosa (a83344) \and Bárbara Cardoso (a80453) \and Pedro Mendes (a79003)}
\date{\today}

\begin{titlepage}

    %título
    \thispagestyle{empty}
    \begin{center}
        \begin{minipage}{0.75\linewidth}
            \centering
            %engenharia logo
            \includegraphics[width=0.4\textwidth]{eng.jpeg}\par\vspace{1cm}
            \vspace{1.5cm}
            %titulos
            \href{https://www.uminho.pt/PT}{\scshape\LARGE Universidade do Minho} \par
            \vspace{1cm}
            \href{https://www.di.uminho.pt/}{\scshape\Large Departamento de Informática} \par
            \vspace{1.5cm}

            \maketitle
        \end{minipage}
    \end{center}

\end{titlepage}

\tableofcontents

\pagebreak

\section{Introdução}

\pagebreak
\section{Descrição do Problema}

\pagebreak
\section{Resolução do Problema}

\pagebreak
\section{Conclusões e Trabalho Futuro}

Este trabalho ajudou a compreender os vários tipos e subtipos de tramas 802.11.

Foi necessário compreender os vários mecanismos de controlo, gestão e transmissão de tramas existentes como por exemplo o scanning passivo e activo e o uso de tramas \textit{RTS} e \textit{CTS} para evitar colisões.


\end{document}

